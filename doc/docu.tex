\documentclass[
10pt,
a4paper
]{scrartcl}

\usepackage[ngerman]{babel}
\usepackage[utf8]{inputenc}
\inputencoding{utf8}
\usepackage{ulem} % underline styles
\usepackage{tikz} % for double underlines
\newcommand{\udensdash}[1]{%
    \tikz[baseline=(todotted.base)]{
        \node[inner sep=1pt,outer sep=1pt] (todotted) {#1};
        \draw[densely dashed] (todotted.south west) -- (todotted.south east);
    }%
}%

%\udensdash{\underline{PrimAndForeignKey}}


\title{DBS Projekt SS2014}
\author{Jan Cortes, Frederic Prackwieser, Franz Rhee}
\date{}

\begin{document}

\maketitle
\tableofcontents

\section*{Einleitung}

\section{1.~Iteration}
Es soll eine Datenbank f\"{u}r die Fußball Bundesliga realisiert werden. Die Datenbank speichert Vereine, Spiele, Spieler, und Ligen. Eine Anwendung stellt vergangene Fußballergebnisse bereit. Spieler sind Vereinen zugeordnet. Vereine sind Ligen zugeordnet. Spiele finden immer zwischen einem Gastgeber und einem Gast statt.
Weiterhin wird die Datenbank für eine Data Mining Anwendung zur Ergebnisprognose genutzt.
Die folgenden Anfragen sollen durch die Anwendung beantwortet werden k\"{o}nnen:

\begin{enumerate}

\item An welchem Tag fand das erste Spiel in dieser Saison statt?
\item Welche Spieler haben in dieser Saison bereits mehr als fünf Tore geschossen?
\item Zeige die Daten aller Spiele an, die am ersten Spieltag aller drei Ligen nach 17 Uhr begonnen haben.
\item Welche Spieler spielen für den Verein “FC Bayern München“? Gib auch die Trikotnummer und das Heimatland jedes Spielers sowie die Anzahl seiner Tore mit aus. Ordne die Ergebnisse aufsteigend nach der Trikotnummer.
\item Wie viele Spiele hat „Hannover 96“ bis heute gewonnen?
\item Gesucht sind Vereinsname, Spieler{\_}ID, Trikotnummer und Name aller Spieler, die für den Verein spielen, der in dieser Saison die meisten Niederlagen erlitten hat (auch mehrere Vereine mit gleicher Anzahl möglich).

\end{enumerate}

\subsection{Anlegen der Datenbank}
Legen Sie eine Datenbank mit dem Namen „bundesliga“ an.

\subsection{Modellierung}
Entwerfen Sie auf Grundlage der Anwendungsbeschreibung und den Daten ein Datenbankschema in, aus der Vorlesung bekannten umgekehrten Chen-Notation mit (min, max) Erweiterung.

\subsection{\"{U}bersetzen ins relationale Modell und SQL}
Schreiben Sie die entsprechenden SQL-Queries zur Erstellung der Tabellen. Achten Sie auf eine gute Wahl von Attributeigenschaften wie NOT NULL, UNIQUE und Schl\"{u}sseln.

\section{2.~Iteration}

\section{3.~Iteration}

\section{4.~Iteration}
\end{document}
